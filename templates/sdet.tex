\documentclass[11pt]{article}       % set main text size
\usepackage[letterpaper,                % set paper size to letterpaper. change to a4paper for resumes outside of North America
top=0.5in,                          % specify top page margin
bottom=0.5in,                       % specify bottom page margin
left=0.5in,                         % specify left page margin
right=0.5in]{geometry}              % specify right page margin
                       
\usepackage{XCharter}               % set font
\usepackage[T1]{fontenc}            % output encoding
\usepackage[utf8]{inputenc}         % input encoding
\usepackage{enumitem}               % enable lists for bullet points: itemize and \item
\usepackage[hidelinks]{hyperref}    % format hyperlinks
\usepackage{titlesec}               % enable section title customization
\raggedright                        % disable text justification
\pagestyle{empty}                   % disable page numbering

% ensure PDF output will be all-Unicode and machine-readable
\input{glyphtounicode}
\pdfgentounicode=1

% format section headings: bolding, size, white space above and below
\titleformat{\section}{\bfseries\large}{}{0pt}{}[\vspace{1pt}\titlerule\vspace{-6.5pt}]

% format bullet points: size, white space above and below, white space between bullets
\renewcommand\labelitemi{$\vcenter{\hbox{\small$\bullet$}}$}
\setlist[itemize]{itemsep=-2pt, leftmargin=12pt}

% resume starts here
\begin{document}

% name
\centerline{\Huge KELLY KEMNITZ}

\vspace{5pt}

% contact information
\centerline{Wichita, KS | (316) 648-6290 | \href{mailto:kellygkemnitz@gmail.com}{kellygkemnitz@gmail.com} | \href{https://www.linkedin.com/in/kellykemnitz/}{linkedin.com/in/kellykemnitz/}}

\vspace{-10pt}

% experience section
\section*{Experience}
\textbf{NetApp} -- Wichita, KS \hfill October 2011 - August 2024 \\
\textbf{Quality Assurance Engineer III} \hfill April 2016 - August 2024 \\
\vspace{-9pt}
\begin{itemize}
  \item Served as team lead, driving process improvements, offering technical guidance, and mentoring junior engineers.
  \item Transitioned iSCSI test configurations to a virtual infrastructure powered by VMware, reducing capital expenditures by 60\%.
  \item Adopted Agile practices within the team, implementing 20-day sprint cycle and daily standups, driving increased productivity and efficiency.
  \item Designed, implemented, and documented a thorough test strategy to ensure seamless compatibility between VMware vSphere ESXi and NVMe over Fabrics (NVMe-oF) storage solutions.
  \item Resolved issues and provided expert guidance by efficiently managing escalations from customer support and technical marketing teams.
  \item Acted as team representative in defect standup meetings, ensuring overall release health and advocated for potential stop-ship issues.
  \item Fostered strong partnerships with Original Equipment Manufacturers (OEMs) to coordinate qualification efforts and develop tailored test plans that aligned with their product release schedules.
\end{itemize}

\textbf{Quality Assurance Engineer II} \hfill October 2014 - April 2016 \\
\vspace{-9pt}
\begin{itemize}
  \item Achieved certification for over 1200 distinct VMware solutions to be used with block storage product offerings, validating their compatibility and performance.
  \item Streamlined VMware deployment times in lab environments by 75\% through the development of PowerCLI scripts and integration of provisioning software Cobbler.
  \item Reduced lab footprint by 40\% through the consolidation of hardware components in test configurations, optimizing resource utilization and maximizing test coverage.
  \item Built and managed a centralized VMware vCenter infrastructure that was shared effectively across development and quality assurance teams.
  \item Cultivated a strategic partnership with VMware to accelerate defect resolution and gain early access to VMware releases, optimizing software development lifecycle.
\end{itemize}

\textbf{Quality Assurance Engineer I} \hfill October 2011 - October 2014 \\
\vspace{-9pt}
\begin{itemize}
  \item Implemented and executed comprehensive test strategy for storage failover driver and block storage product, resulting in over 50 unique solutions certified for Windows Server Catalog.
  \item Developed documentation for Windows Server solutions, enhancing the setup and configuration processes for internal and external customers.
  \item Identified and validated use cases for Hyper-V Virtual Fibre Channel, enabling enhanced storage scalability.
\end{itemize}

% \textbf{SCTelcom} -- Medicine Lodge, KS \hfill October 2003 - August 2008 \\
% \textbf{Network Administrator II} \hfill January 2005 - August 2008 \\
% \vspace{-9pt}
% \begin{itemize}
%   \item Piloted a managed IT services initiative, helping local small businesses enhance their technology infrastructure.
%   \item Maintained and provided administration of system servers and networking equipment for internal use.
%   \item Provisioned DSL, wireless, and VoIP circuits for external customers.
%   \item Compiled an inventory of datacom equipment and sold obsolete items to hardware vendors.
% \end{itemize}

% \textbf{Outside Plant Technician} \hfill October 2003 - January 2005 \\
% \vspace{-9pt}
% \begin{itemize}
%   \item Installed telephone and data services for residential and business customers.
%   \item Served as on-call technician every two weeks, providing 24/7 availability for residential and business customers experiencing telephone outages.
% \end{itemize}

\vspace{-18.5pt}

% projects section
\section*{Projects}
\textbf{Flask Resume App} \hfill \href{https://github.com/kellygkemnitz/flask-resume-app}{github.com/kellygkemnitz/flask-resume-app} \\
\vspace{-9pt}
\begin{itemize}
 \item Simple Flask application that serves a PDF file and displays it directly on a homepage.
\end{itemize}
%
%\textbf{Project Title} \hfill \href{https://mitcommlab.mit.edu/meche/commkit/portfolio/}{project.com} \\
%\vspace{-9pt}
%\begin{itemize}
%  \item Only list real projects, not mandatory school projects or trivial tutorial projects found online 
%  \item Something that someone uses to solve a problem
%  \item Something that has users (can be just you, as long as you use it often) and is actively maintained and not just rotting in a GitHub repo, never to see a PR for the rest of its life
%\end{itemize}
%
%\textbf{Project Title} \hfill \href{https://www.hardwareishard.net/portfolio-database}{blog.com/projectname} \\
%\vspace{-9pt}
%\begin{itemize}
%  \item Resume checklist: \url{https://old.reddit.com/r/EngineeringResumes/wiki/checklist}
%  \item Google Docs version of this template: \url{https://docs.google.com/document/d/1MBvhATv8y-ESORopRoLSZ3f3HjkM_Qa_f8fIHAEqgnI/edit}
%\end{itemize}

% alternate formatting for projects section
% \section*{Projects}
% \begin{itemize}
%   \item \textbf{QuantSoftware Toolkit:} Open source Python library for data analysis and machine learning for finance
%   \item \textbf{GitHub Visualization:} Data visualization of Git Log data using D3 to analyze project trends over time
%   \item \textbf{Recommendation System:} Music and movie recommender systems using collaborative filtering on public datasets
%   \item \textbf{Mac Setup:} Book that gives step by step instructions on setting up developer environment on Mac OS
% \end{itemize}

\vspace{-18.5pt}

% education section
\section*{Education}
\textbf{\href{https://www.wichita.edu}{Wichita State University}} -- BS in Computer Science \hfill August 2008 - May 2012


%\vspace{-18.5pt}


% skills section
\section*{Skills}
\textbf{Developer Tools:} Git, BitBucket, VS Code, JupyterLab \\
\textbf{Frameworks:} Ansible, pytest, Docker \\
\textbf{Languages:} VMware PowerCLI, PowerShell, Python, Bash \\
\textbf{Networking:} Cisco IOS, Cisco NX-OS, NVIDIA Mellanox MLNX-OS \\
\textbf{Operating Systems:} VMware vSphere ESXi, Microsoft Windows Server, Red Hat Enterprise Linux, SUSE Linux Enterprise Server, Oracle Linux, Ubuntu \\
\textbf{Project Management Tools:} Jira, Confluence \\
\textbf{Storage Protocols:} Fibre Channel, iSCSI, NFS, NVMe over Fabrics (NVMe-oF), Serial Attached SCSI (SAS)

\vspace{-6.5pt}


\end{document}